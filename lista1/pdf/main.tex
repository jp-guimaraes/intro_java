\documentclass[12pt]{article} 
\usepackage[brazil]{babel} 
\usepackage[utf8]{inputenc}

\usepackage{listings}
\usepackage{color}

\definecolor{dkgreen}{rgb}{0,0.6,0}
\definecolor{gray}{rgb}{0.5,0.5,0.5}
\definecolor{mauve}{rgb}{0.58,0,0.82}

\lstset{frame=tb,
  language=Java,
  aboveskip=3mm,
  belowskip=3mm,
  showstringspaces=false,
  columns=flexible,
  basicstyle={\small\ttfamily},
  numbers=none,
  numberstyle=\tiny\color{gray},
  keywordstyle=\color{blue},
  commentstyle=\color{dkgreen},
  stringstyle=\color{mauve},
  breaklines=true,
  breakatwhitespace=true,
  tabsize=3
}


\begin{document} 

\title{Lista de exercícios 1} 
\author{João Paulo F. Guimarães \\ joao.guimaraes@ifrn.edu.br} 
\date{novembro de 2019} 

\maketitle %cria o título

\section{Introdução}

Todas as questões devem utilizar a linguagem de programação Java.

\section{Revisão}

\begin{lstlisting}
// Hello.java
import javax.swing.JApplet;
import java.awt.Graphics;

public class Hello extends JApplet {
    public void paintComponent(Graphics g) {
        g.drawString("Hello, world!", 65, 95);
    }    
}
\end{lstlisting}


\section{Questões}

\begin{enumerate}

	\item Escreva um programa em Java que calcule a média entre duas notas.



\end{enumerate}



% \begin{thebibliography}{99} % Até 99 referencias
% \bibitem{wiki} % este é o nome que é usado para citar a referencia
% http://en.wikipedia.org/ % referência

% \end{thebibliography}

\end{document}