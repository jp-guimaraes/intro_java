\documentclass[12pt]{article} 
\usepackage[brazil]{babel} 
\usepackage[utf8]{inputenc}


\usepackage{listings}
\usepackage{color}

\definecolor{dkgreen}{rgb}{0,0.6,0}
\definecolor{gray}{rgb}{0.5,0.5,0.5}
\definecolor{mauve}{rgb}{0.58,0,0.82}

\lstset{frame=tb,
  language=Java,
  aboveskip=3mm,
  belowskip=3mm,
  showstringspaces=false,
  columns=flexible,
  basicstyle={\small\ttfamily},
  numbers=none,
  numberstyle=\tiny\color{gray},
  keywordstyle=\color{blue},
  commentstyle=\color{dkgreen},
  stringstyle=\color{mauve},
  breaklines=true,
  breakatwhitespace=true,
  tabsize=3
}


\begin{document} 

\title{Lista de exercícios 1} 
\author{João Paulo F. Guimarães \\ joao.guimaraes@ifrn.edu.br} 
\date{novembro de 2019} 

\maketitle %cria o título

\section{Instruções}

\begin{itemize}

  \item Todas as questões devem utilizar a linguagem de programação Java.

  \item Cada questão deve ter seu próprio código independente.

  \item O repositório contém alguns exemplos de códigos funcionais que ajudam na resolução da lista. A maioria das questões podem ser resolvidas alterando ou combinando esses códigos. Cada um deles é detalhado na seção a seguir chamada Revisão.

  \item Para submeter a lista para avaliação basta criar um único arquivo .rar ou .zip contendo todos os códigos implementados. Eles devem ser submetidos via SUAP.
  \end{itemize}


\section{Revisão}

Dentro do repositório existe uma pasta denominada exemplos. Nela pode-se encontrar alguns códigos básicos mostrando a sintaxe de Java.

\subsection{Exemplo 1 - Código Olá mundo}

No primeiro exemplo temos o famoso olá mundo, mostrando como imprimir no terminal.

% \begin{lstlisting}
% // Hello.java
% public class ex1 {
% 	public static void main(String[ ] args) {
%         System.out.println("Ola mundo!");
%     }
% }
% \end{lstlisting}

\subsection{Exemplo 2 - Entrada de dados}

O segundo exemplo mostra como fazer entrada de dados pelo terminal para variáveis dentro do seu programa.

% \begin{lstlisting}
% // Hello.java
% public class ex1 {
% 	public static void main(String[ ] args) {
%         System.out.println("Ola mundo!");
%     }
% }
% \end{lstlisting}


\subsection{Exemplo 3 - Calculando uma média}

O terceiro exemplo mostra como manipular duas variáveis que armazenam duas notas para calcular a média entre elas.

% \begin{lstlisting}
% public class exemplo_03 {
%   public static void main(String[] args) {
%         System.out.println("Ola mundo, començando os testes com variaveis!");
        
%         double nota1 = 8.5;
%         double nota2 = 7.2;

%         System.out.println("Notas informadas: ");
%         System.out.println(nota1);
%         System.out.println("e");
%         System.out.println(nota2);        
        
%         double media = (nota1+nota2)/2;

%         System.out.println("Média calculada: ");
%         System.out.println(media);       
%     }
% }


% \end{lstlisting}



\subsection{Exemplo 4 - Controle de fluxo}

O quarto exemplo  mostra como manipular duas variáveis que armazenam duas notas para calcular a média entre elas.

% \begin{lstlisting}
% public class exemplo_03 {
%   public static void main(String[] args) {
%         System.out.println("Olá mundo, començando os testes com controle de fluxo, estruturas de decisão!");
        
%         double nota1 = 8.5;
%         double nota2 = 7.2;

%         System.out.println("Notas informadas: ");
%         System.out.println(nota1);
%         System.out.println("e");
%         System.out.println(nota2);         
               
%         double media = (nota1+nota2)/2;

%         System.out.println("Média calculada: ");
%         System.out.println(media);              
        
%         boolean teste = (media >= 6.0);
        
%         // Imprimindo o valor do booleano
%         System.out.println(teste);
        
%         if(teste) {
%           System.out.println("Aprovado!");
%         }
%         else {
%           System.out.println("Reprovado!");
%         }
        
%         System.out.println("Fim!");
%     }
% }


% \end{lstlisting}



\section{Questões}

\begin{enumerate}
  \item Crie um programa em Java que calcule a média entre três notas.
  \item Crie um programa que peça duas notas ao usuário e calcule a média entre elas.
  \item Faça um programa que pergunte a idade ao usuário. Processe essa entrada e diga se o usuário é maior de idade ou não.
  \item Faça um programa que calcule a área de um quadrado de lado L. O tamanho L deve ser pedido ao usuário.
  \item Crie um programa que, dado duas notas fornecidas pelo usuário, diga se ele foi APROVADO (média $>=$ 6), se ele está REPROVADO (média $<3$), ou em recuperação (caso contrário).
  \item Faça um programa que calcula a média ponderada entre 3 notas. Tanto as notas quanto os pesos das notas devem ser pedidos ao usuário.
\end{enumerate}



% \begin{thebibliography}{99} % Até 99 referencias
% \bibitem{wiki} % este é o nome que é usado para citar a referencia
% http://en.wikipedia.org/ % referência

% \end{thebibliography}

\end{document}