\documentclass[12pt]{article} 
\usepackage[brazil]{babel} 
\usepackage[utf8]{inputenc}

\begin{document} 

\title{Lista de exercícios 1} 
\author{João Paulo F. Guimarães \\ joao.guimaraes@ifrn.edu.br} 
\date{novembro de 2019} 

\maketitle %cria o título

\section{Introdução}

Todas as questões devem utilizar a linguagem de programação Java.

\section{Revisão}



\section{Questões}

\begin{enumerate}

	\item Escreva um programa em Java que calcule a média entre duas notas.



\end{enumerate}



\begin{thebibliography}{99} % Até 99 referencias
\bibitem{wiki} % este é o nome que é usado para citar a referencia
http://en.wikipedia.org/ % referência
\bibitem{stern}
http://www.ime.usp.br/$\sim$ jstern

\end{thebibliography}

\end{document} %fim do documento. O que vem depois daqui não é gerado

